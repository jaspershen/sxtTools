\documentclass[]{article}
\usepackage{lmodern}
\usepackage{amssymb,amsmath}
\usepackage{ifxetex,ifluatex}
\usepackage{fixltx2e} % provides \textsubscript
\ifnum 0\ifxetex 1\fi\ifluatex 1\fi=0 % if pdftex
  \usepackage[T1]{fontenc}
  \usepackage[utf8]{inputenc}
\else % if luatex or xelatex
  \ifxetex
    \usepackage{mathspec}
  \else
    \usepackage{fontspec}
  \fi
  \defaultfontfeatures{Ligatures=TeX,Scale=MatchLowercase}
\fi
% use upquote if available, for straight quotes in verbatim environments
\IfFileExists{upquote.sty}{\usepackage{upquote}}{}
% use microtype if available
\IfFileExists{microtype.sty}{%
\usepackage{microtype}
\UseMicrotypeSet[protrusion]{basicmath} % disable protrusion for tt fonts
}{}
\usepackage[margin=1in]{geometry}
\usepackage{hyperref}
\hypersetup{unicode=true,
            pdftitle={R语言stringr包的使用},
            pdfborder={0 0 0},
            breaklinks=true}
\urlstyle{same}  % don't use monospace font for urls
\usepackage{color}
\usepackage{fancyvrb}
\newcommand{\VerbBar}{|}
\newcommand{\VERB}{\Verb[commandchars=\\\{\}]}
\DefineVerbatimEnvironment{Highlighting}{Verbatim}{commandchars=\\\{\}}
% Add ',fontsize=\small' for more characters per line
\usepackage{framed}
\definecolor{shadecolor}{RGB}{248,248,248}
\newenvironment{Shaded}{\begin{snugshade}}{\end{snugshade}}
\newcommand{\AlertTok}[1]{\textcolor[rgb]{0.94,0.16,0.16}{#1}}
\newcommand{\AnnotationTok}[1]{\textcolor[rgb]{0.56,0.35,0.01}{\textbf{\textit{#1}}}}
\newcommand{\AttributeTok}[1]{\textcolor[rgb]{0.77,0.63,0.00}{#1}}
\newcommand{\BaseNTok}[1]{\textcolor[rgb]{0.00,0.00,0.81}{#1}}
\newcommand{\BuiltInTok}[1]{#1}
\newcommand{\CharTok}[1]{\textcolor[rgb]{0.31,0.60,0.02}{#1}}
\newcommand{\CommentTok}[1]{\textcolor[rgb]{0.56,0.35,0.01}{\textit{#1}}}
\newcommand{\CommentVarTok}[1]{\textcolor[rgb]{0.56,0.35,0.01}{\textbf{\textit{#1}}}}
\newcommand{\ConstantTok}[1]{\textcolor[rgb]{0.00,0.00,0.00}{#1}}
\newcommand{\ControlFlowTok}[1]{\textcolor[rgb]{0.13,0.29,0.53}{\textbf{#1}}}
\newcommand{\DataTypeTok}[1]{\textcolor[rgb]{0.13,0.29,0.53}{#1}}
\newcommand{\DecValTok}[1]{\textcolor[rgb]{0.00,0.00,0.81}{#1}}
\newcommand{\DocumentationTok}[1]{\textcolor[rgb]{0.56,0.35,0.01}{\textbf{\textit{#1}}}}
\newcommand{\ErrorTok}[1]{\textcolor[rgb]{0.64,0.00,0.00}{\textbf{#1}}}
\newcommand{\ExtensionTok}[1]{#1}
\newcommand{\FloatTok}[1]{\textcolor[rgb]{0.00,0.00,0.81}{#1}}
\newcommand{\FunctionTok}[1]{\textcolor[rgb]{0.00,0.00,0.00}{#1}}
\newcommand{\ImportTok}[1]{#1}
\newcommand{\InformationTok}[1]{\textcolor[rgb]{0.56,0.35,0.01}{\textbf{\textit{#1}}}}
\newcommand{\KeywordTok}[1]{\textcolor[rgb]{0.13,0.29,0.53}{\textbf{#1}}}
\newcommand{\NormalTok}[1]{#1}
\newcommand{\OperatorTok}[1]{\textcolor[rgb]{0.81,0.36,0.00}{\textbf{#1}}}
\newcommand{\OtherTok}[1]{\textcolor[rgb]{0.56,0.35,0.01}{#1}}
\newcommand{\PreprocessorTok}[1]{\textcolor[rgb]{0.56,0.35,0.01}{\textit{#1}}}
\newcommand{\RegionMarkerTok}[1]{#1}
\newcommand{\SpecialCharTok}[1]{\textcolor[rgb]{0.00,0.00,0.00}{#1}}
\newcommand{\SpecialStringTok}[1]{\textcolor[rgb]{0.31,0.60,0.02}{#1}}
\newcommand{\StringTok}[1]{\textcolor[rgb]{0.31,0.60,0.02}{#1}}
\newcommand{\VariableTok}[1]{\textcolor[rgb]{0.00,0.00,0.00}{#1}}
\newcommand{\VerbatimStringTok}[1]{\textcolor[rgb]{0.31,0.60,0.02}{#1}}
\newcommand{\WarningTok}[1]{\textcolor[rgb]{0.56,0.35,0.01}{\textbf{\textit{#1}}}}
\usepackage{graphicx,grffile}
\makeatletter
\def\maxwidth{\ifdim\Gin@nat@width>\linewidth\linewidth\else\Gin@nat@width\fi}
\def\maxheight{\ifdim\Gin@nat@height>\textheight\textheight\else\Gin@nat@height\fi}
\makeatother
% Scale images if necessary, so that they will not overflow the page
% margins by default, and it is still possible to overwrite the defaults
% using explicit options in \includegraphics[width, height, ...]{}
\setkeys{Gin}{width=\maxwidth,height=\maxheight,keepaspectratio}
\IfFileExists{parskip.sty}{%
\usepackage{parskip}
}{% else
\setlength{\parindent}{0pt}
\setlength{\parskip}{6pt plus 2pt minus 1pt}
}
\setlength{\emergencystretch}{3em}  % prevent overfull lines
\providecommand{\tightlist}{%
  \setlength{\itemsep}{0pt}\setlength{\parskip}{0pt}}
\setcounter{secnumdepth}{0}
% Redefines (sub)paragraphs to behave more like sections
\ifx\paragraph\undefined\else
\let\oldparagraph\paragraph
\renewcommand{\paragraph}[1]{\oldparagraph{#1}\mbox{}}
\fi
\ifx\subparagraph\undefined\else
\let\oldsubparagraph\subparagraph
\renewcommand{\subparagraph}[1]{\oldsubparagraph{#1}\mbox{}}
\fi

%%% Use protect on footnotes to avoid problems with footnotes in titles
\let\rmarkdownfootnote\footnote%
\def\footnote{\protect\rmarkdownfootnote}

%%% Change title format to be more compact
\usepackage{titling}

% Create subtitle command for use in maketitle
\providecommand{\subtitle}[1]{
  \posttitle{
    \begin{center}\large#1\end{center}
    }
}

\setlength{\droptitle}{-2em}

  \title{R语言stringr包的使用}
    \pretitle{\vspace{\droptitle}\centering\huge}
  \posttitle{\par}
    \author{}
    \preauthor{}\postauthor{}
    \date{}
    \predate{}\postdate{}
  

\begin{document}
\maketitle

\hypertarget{r-for-data-sciencestringr}{%
\subsubsection{\texorpdfstring{最近还在适应美帝的生后,并且因为还没有开始新的课题,所以晚上时间会比较充裕一些,一直在看\textbf{R
for data
science}这本书,前面一直没有记笔记,后面觉得还是需要养成记笔记的习惯。所以从今天开始,即使每天少看一些,也要记下来自己的读书笔记。今天就从字符串处理stringr包开始,也是第一部分。}{最近还在适应美帝的生后,并且因为还没有开始新的课题,所以晚上时间会比较充裕一些,一直在看R for data science这本书,前面一直没有记笔记,后面觉得还是需要养成记笔记的习惯。所以从今天开始,即使每天少看一些,也要记下来自己的读书笔记。今天就从字符串处理stringr包开始,也是第一部分。}}\label{r-for-data-sciencestringr}}

需要先加载包

\begin{Shaded}
\begin{Highlighting}[]
\KeywordTok{library}\NormalTok{(tidyverse)}
\end{Highlighting}
\end{Shaded}

\begin{verbatim}
## -- Attaching packages ---------------------------------------------------------- tidyverse 1.2.1 --
\end{verbatim}

\begin{verbatim}
## v ggplot2 3.1.1       v purrr   0.3.2  
## v tibble  2.1.1       v dplyr   0.8.0.1
## v tidyr   0.8.3       v stringr 1.4.0  
## v readr   1.3.1       v forcats 0.4.0
\end{verbatim}

\begin{verbatim}
## -- Conflicts ------------------------------------------------------------- tidyverse_conflicts() --
## x dplyr::filter() masks stats::filter()
## x dplyr::lag()    masks stats::lag()
\end{verbatim}

\begin{Shaded}
\begin{Highlighting}[]
\KeywordTok{library}\NormalTok{(stringr)}
\end{Highlighting}
\end{Shaded}

\hypertarget{string-lenght}{%
\subsubsection{1 string lenght}\label{string-lenght}}

如何查看一个字符串中字符的长度

\begin{Shaded}
\begin{Highlighting}[]
\KeywordTok{str_length}\NormalTok{(}\KeywordTok{c}\NormalTok{(}\StringTok{"a"}\NormalTok{, }\StringTok{"R for data science"}\NormalTok{, }\OtherTok{NA}\NormalTok{))}
\end{Highlighting}
\end{Shaded}

\begin{verbatim}
## [1]  1 18 NA
\end{verbatim}

\hypertarget{section}{%
\subsubsection{2 合并字符串}\label{section}}

\begin{Shaded}
\begin{Highlighting}[]
\KeywordTok{str_c}\NormalTok{(}\StringTok{"x"}\NormalTok{, }\StringTok{"y"}\NormalTok{)}
\end{Highlighting}
\end{Shaded}

\begin{verbatim}
## [1] "xy"
\end{verbatim}

\begin{Shaded}
\begin{Highlighting}[]
\KeywordTok{str_c}\NormalTok{(}\StringTok{"x"}\NormalTok{, }\StringTok{"y"}\NormalTok{, }\StringTok{"z"}\NormalTok{)}
\end{Highlighting}
\end{Shaded}

\begin{verbatim}
## [1] "xyz"
\end{verbatim}

同样可以使用\emph{sep}参数来控制合并时中间分隔符。

\begin{Shaded}
\begin{Highlighting}[]
\KeywordTok{str_c}\NormalTok{(}\StringTok{"x"}\NormalTok{, }\StringTok{"y"}\NormalTok{, }\DataTypeTok{sep =} \StringTok{","}\NormalTok{)}
\end{Highlighting}
\end{Shaded}

\begin{verbatim}
## [1] "x,y"
\end{verbatim}

和很多其他函数一样,NA值是一个问题,比如下面的例子,这时候可以使用\texttt{str\_replace\_na()}函数来处理:

\begin{Shaded}
\begin{Highlighting}[]
\NormalTok{x <-}\StringTok{ }\KeywordTok{c}\NormalTok{(}\StringTok{"abc"}\NormalTok{, }\OtherTok{NA}\NormalTok{)}
\KeywordTok{str_c}\NormalTok{(}\StringTok{"|-"}\NormalTok{, x, }\StringTok{"-|"}\NormalTok{)}
\end{Highlighting}
\end{Shaded}

\begin{verbatim}
## [1] "|-abc-|" NA
\end{verbatim}

\begin{Shaded}
\begin{Highlighting}[]
\KeywordTok{str_c}\NormalTok{(}\StringTok{"|-"}\NormalTok{, }\KeywordTok{str_replace_na}\NormalTok{(x), }\StringTok{"-|"}\NormalTok{)}
\end{Highlighting}
\end{Shaded}

\begin{verbatim}
## [1] "|-abc-|" "|-NA-|"
\end{verbatim}

同样从上面的例子也可以看到,\texttt{str\_c}是一个vector函数,他会将长度较短的对象循环,使其与长度最长的对象长度一样。

\begin{Shaded}
\begin{Highlighting}[]
\KeywordTok{str_c}\NormalTok{(}\StringTok{"prefix-"}\NormalTok{, }\KeywordTok{c}\NormalTok{(}\StringTok{"a"}\NormalTok{, }\StringTok{"b"}\NormalTok{, }\StringTok{"c"}\NormalTok{), }\StringTok{"-suffix"}\NormalTok{)}
\end{Highlighting}
\end{Shaded}

\begin{verbatim}
## [1] "prefix-a-suffix" "prefix-b-suffix" "prefix-c-suffix"
\end{verbatim}

下面要简单介绍一下在\texttt{str\_c}中非常有用的两个参数,\texttt{sep}和\texttt{collapse}。其中\texttt{collapse}非常形象,\texttt{坍塌},也就是将一个向量中的多个对象合并为一个字符。可以看下面的例子:

\begin{Shaded}
\begin{Highlighting}[]
\KeywordTok{str_c}\NormalTok{(}\KeywordTok{c}\NormalTok{(}\StringTok{"x"}\NormalTok{, }\StringTok{"y"}\NormalTok{, }\StringTok{"z"}\NormalTok{), }\DataTypeTok{collapse =} \StringTok{", "}\NormalTok{)}
\end{Highlighting}
\end{Shaded}

\begin{verbatim}
## [1] "x, y, z"
\end{verbatim}

\begin{Shaded}
\begin{Highlighting}[]
\KeywordTok{str_c}\NormalTok{(}\KeywordTok{c}\NormalTok{(}\StringTok{"x"}\NormalTok{, }\StringTok{"y"}\NormalTok{, }\StringTok{"z"}\NormalTok{), }\DataTypeTok{sep =} \StringTok{", "}\NormalTok{)}
\end{Highlighting}
\end{Shaded}

\begin{verbatim}
## [1] "x" "y" "z"
\end{verbatim}

\hypertarget{section-1}{%
\subsubsection{3 提取字字符}\label{section-1}}

有时候我们需要一个字符串中的某部分,这时候可以使用\texttt{str\_sub()}。

\begin{Shaded}
\begin{Highlighting}[]
\NormalTok{x <-}\StringTok{ }\KeywordTok{c}\NormalTok{(}\StringTok{"Apple"}\NormalTok{, }\StringTok{"Banana"}\NormalTok{, }\StringTok{"Pear"}\NormalTok{)}
\KeywordTok{str_sub}\NormalTok{(}\DataTypeTok{string =}\NormalTok{ x, }\DataTypeTok{start =} \DecValTok{1}\NormalTok{, }\DataTypeTok{end =} \DecValTok{3}\NormalTok{)}
\end{Highlighting}
\end{Shaded}

\begin{verbatim}
## [1] "App" "Ban" "Pea"
\end{verbatim}

该函数一共有三个参数,start和end分别是指从哪开始和结束提取字符串。
还可以反过来提取字符串,也就是将参数设置为负数。

\begin{Shaded}
\begin{Highlighting}[]
\KeywordTok{str_sub}\NormalTok{(x, }\DecValTok{-3}\NormalTok{, }\DecValTok{-1}\NormalTok{)}
\end{Highlighting}
\end{Shaded}

\begin{verbatim}
## [1] "ple" "ana" "ear"
\end{verbatim}

\hypertarget{section-2}{%
\subsubsection{4 区域设置}\label{section-2}}

对于某些规定,不同国家或者地区不太一样,比如排序,默认英语里面的排序是按照英语字母顺序,但是不同地方可能不一样,比如下面的例子:

\begin{Shaded}
\begin{Highlighting}[]
\NormalTok{x <-}\StringTok{ }\KeywordTok{c}\NormalTok{(}\StringTok{"apple"}\NormalTok{, }\StringTok{"eggplant"}\NormalTok{, }\StringTok{"banana"}\NormalTok{)}
\KeywordTok{str_sort}\NormalTok{(x, }\DataTypeTok{locale =} \StringTok{"en"}\NormalTok{)  }\CommentTok{# English}
\end{Highlighting}
\end{Shaded}

\begin{verbatim}
## [1] "apple"    "banana"   "eggplant"
\end{verbatim}

\begin{Shaded}
\begin{Highlighting}[]
\KeywordTok{str_sort}\NormalTok{(x, }\DataTypeTok{locale =} \StringTok{"haw"}\NormalTok{) }\CommentTok{# Hawaiian}
\end{Highlighting}
\end{Shaded}

\begin{verbatim}
## [1] "apple"    "eggplant" "banana"
\end{verbatim}

感觉这个功能不会太常用。了解一下就好了。


\end{document}
